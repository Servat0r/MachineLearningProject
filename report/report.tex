\documentclass[12pt]{article}
\usepackage[utf8]{inputenc}
\usepackage{cleveref}
\usepackage{graphicx}
\usepackage{subfig}
\usepackage{geometry}

\geometry{a4paper, left=30mm, right=30mm}

\setlength\parindent{0pt}

\title{Neural Network Implementation}
\author{Salvatore Correnti, \texttt{s.correnti@studenti.unipi.it}\\
  Alberto L'Episcopo, \texttt{a.lepiscopo1@studenti.unipi.it}\\
  Gaetano Nicassio, \texttt{g.nicassio??@studenti.unipi.it}
  \and Project A of ML course (654AA), A.Y. 2022-23}
\date{31th December 2022}

\begin{document}
\maketitle

\begin{abstract}
  The aim of this report is to illustrate the implementation and testing of a framework for creation and usage of basic Neural Networks.
  We have implemented this framework in Python by exploiting the NumPy numerical library, with a focus on modularity and extensibility of the code.
  We have used the MONK and the ML-CUP22 datasets for testing this framework against respectively classification and regression problems.
  Model selection is implemented as a Grid Search with Holdout and K-fold as validation techniques.
 
\end{abstract}

\section{Introduction}

Our goal is to realize a Neural Network model simulator and apply it to the given problem sets: the 3 monks classification problems and the ML cup regression, trying of course to make it as much predictive as possible over the test set.

We implemented a feed-forward multi-layer perceptron (actually used just one hidden layer) supporting standard back propagation and several versions of gradient descent: multi-batched, single-batched, with or without momentum and employing various loss and activation functions.

Our main assumption to predict the ML cup blind test values was that the underlying function was smooth and compact supported, so that we could expect to really learn it using a standard fully connected MLP. Moreover observing the data and trying to fit them with various models we realized that output data had been perturbed by some noise and then we paid careful attention not to commit overfitting. 

\section{Method}

\subsection{Code}

We implemented our NN in \texttt{Python 2.7.13} and relied on the \texttt{numpy} library for linear algebra operations that are paramount in back propagation algorithm. We put some effort in keeping our code as much parametric as possible in those features that we deemed as relevant for the model definition so that we would have been able to adapt it along the way.

\paragraph{}
In file \texttt{network.py} you can find the class Network that is basically a list of Layer class objects and a pair of activation functions (one internal and one for the last layer, often required to be different). This class implements feed-forward and propagate-back methods exploiting the homonym Layer methods as building blocks, thus enhancing code modularity. The Layer class is composed of a numpy matrix, a bias array, an activation function and a parameter devoted to regularization. We chose to initialize layers using a normal distribution of mean 0 and variance 1 for each entry as advised in literature although different choices (like flat 0-symmetric distributions) did not make any real differences in terms of accuracy or loss.

\paragraph{}
In file \texttt{gradient\_descent.py} we implemented \texttt{gradient\_descent} procedure keeping it parametric in batch size so that we could make more trials over different batch sizes. We implemented also momentum heuristic keeping it parametric in $\beta$ so that we could tune it or even shut it down during our model development.
Finally we chose to regularize our model adding to the loss function the squared norm of weights multiplied by $\lambda$ (\texttt{mu} in code, since lambda is a python keyword) as a penalty term. As far as stopping criterion is concerned at a first glance we deferred its implementation and then found it unworthy since our learning plots always stabilized after 2000 epochs, hence it was sufficient to manually insert that threshold.

\paragraph{}
In file \texttt{utility.py} we defined a class DiffFunction representing a function and its derivative, so that we could easily initialize our networks. Then we coded 4 different activation functions (\texttt{tanh}, \texttt{softMax}, \texttt{softPlus}, \texttt{reLU}, \texttt{identity}) and 4 different loss functions (\texttt{crossEntropy}, \texttt{binaryCrossEntropy}, \texttt{euclideanLoss}, \texttt{squaredLoss}) belonging to that class. At the end of the file you can find a couple of accuracy evaluators and some I/O formatting related procedures employed in experiments.

\subsection{Preprocessing}

Our first attempt did not encode monks inputs one-hot and it was way harder to achieve the 100\% accuracy, easily conquered later exploiting the suggested preprocessing. We analyzed cup input data finding that they were normalized yet, indeed they are distributed exactly as a normal distribution of mean 0 and variance 1, so we did not apply any transformation. Cup output was always less than 30 in norm thus we regarded a normalization as useless. Only our last model employed a little input preprocessing: we fed not only inputs but even their pairwise products. Apart from that we did not employ any tricky preprocessing as in monks case.

\subsection{Validation Scheme}

Monks dataset was endowed of an explicit test set to finally assess the model performance, we just separated a random fraction (25\%) of the training set for validation purposes. In cup dataset instead we needed to randomly separate an internal test set (10\%) by ourselves, thus you can find 2 different files attached: cup.test and cup.train. We separated them in the beginning in order to keep test strictly apart. During our cross validation we further split the training set between train (75\%) and validation (25\%). It is worth noting that we did not employ a k-fold approach but sticked to sample a random validation for every trial, then we took the average loss (or accuracy) over those sampling (usually 10, since variance was low enough).

\subsection{Trials Pursued}

Monks started working with 100\% accuracy immediately after adopting one-hot encoding, we had just to pay some attention to regularization for the 3rd set.

\paragraph{}
Cup dataset was trickier and it required us to try the heuristics described above (and in deeper details in experiments section) first manually trying to guess the right order of magnitude for parameters and then with a more exhaustive grid search. Once found the best possible NN model we tried to slightly modify our approach trying to filter out the noise: we trained a batch of 20 networks separately and ranked them for increasing loss on validation, took the bests and averaged their feed-forward results on test set assessing that this performed slightly better than each one taken separately. In the end we tried the pairwise products input preprocessing, combined with the previous heuristic, finding that it performed again slightly better. 


%%% Local Variables:
%%% mode: latex
%%% TeX-master: "report"
%%% End:

\section{Experiments}

\subsection{MONK's results}

\subsubsection{Architecture and hyper-parameters}
We have used a fully connected NN with layers of size 17-15-2, with a \texttt{tanh} activation function in the hidden layer, and a \texttt{softmax} for the output layer.

We chose \texttt{crossEntropy} as the loss function, i.e. if the output of the NN was $y=(y_1,y_2)$ and the label was $\hat y=(\hat y_1,\hat y_2)$, then $$L(y,\hat y)=-\hat y_1\log y_1-\hat y_2\log y_2$$

We used mini-batch gradient descent with $\texttt{batch\_size}=10$ and momentum with parameter $\beta$.

We also used L2 regularization with parameter $\lambda$.

Our choice of hyper-parameters is shown in \cref{fig:hyper} below, along with the performance.

\begin{figure}[h]
    \begin{tabular}{|l|c|c|c|c|c|}
        \hline 
        Task & $\eta$ & $\lambda$ & $\beta$ & Loss (train/valid) & Accuracy (train/test) \\ \hline
        MONK 1 & 0.1 & 0.003 & 0.8 & 0.0466/0.0552  & 100\%/100\% \\ \hline
        MONK 2 & 0.1 & 0.003 & 0.8 & 0.0531/0.0896 & 100\%/100\% \\ \hline
        MONK 3 (reg) & 0.01 & 0.02 & 0.9 & 0.2116/0.4599 & 94.2\%/97.2\% \\ \hline
        MONK 3 (non reg) & 0.1 & 0.0001 & 0.9 & 0.0052/0.7687 & 100\%/93.9\% \\ \hline
    \end{tabular}
    \caption{Performance results on MONK dataset}
    \label{fig:hyper}
\end{figure}


\subsubsection{Learning curves}

Now we show the learning curves of the three MONK tasks, plotting loss and accuracy over the training set and the validation set.

\begin{figure}
    \centering
    \subfloat[MONK 1]{\includegraphics[width=0.5\textwidth]{monks1}}
    \subfloat[MONK 2]{\includegraphics[width=0.5\textwidth]{monks2}}
    \caption{Learning curves on MONK 1 \& 2}
    \label{fig:monk12}
\end{figure}

\begin{figure}
    \centering    \subfloat[regularized\label{fig:monk3}]{\includegraphics[width=0.5\textwidth]{monks3}}
    \subfloat[not regularized\label{fig:monk3-nr}]{\includegraphics[width=0.5\textwidth]{monks3-nr}}
    \caption{Learning curves for MONK 3}
\end{figure}

We notice that the training loss of MONK 3 (\cref{fig:monk3}) is quite high: this is because the training set has some noise and hence we are keeping a quite strong regularization.
If instead we remove the regularization, then it overfits as shown in \cref{fig:monk3-nr}

\clearpage

\subsection{CUP results}

The first thing we did was to take 100 random data-points from the training set and put them in a separate file for internal testing. So we had 916 data-points for the actual training and validation.

For every training instance we randomly split the internal training set at random for 75\% actual training and 25\% validation. We always used mini-batch gradient descent with $\texttt{batch\_size}=32$.


\subsubsection{Preliminary trials}

We first tried to see if we needed a deep NN, but increasing the number of layers didn't yield significant improvements.

Then we chose to use MEE as the loss function instead of MSE, because it was the metric we actually strove to minimize.

So we wanted a model with layers 10-H-2 for some $H\in\{ 10, 20, 50 \}$, and an activation function in the hidden layer among \texttt{tanh}, \texttt{sigmoid}, \texttt{reLU} and \texttt{softMax}.

Then we had also to tune the hyper-parameters $\eta,\lambda,\beta$ so we performed a grid search on these 5 parameters, distributing the load among many machines.

\subsubsection{Grid search}

For each choice of an activation function and a number of hidden neurons $H$ we made a different machine run a grid search among these values:
\begin{itemize}
    \item $\eta\in\{ 0.1, 0.05, 0.01, 0.005, 0.001 \}$
    \item $\lambda\in\{ 0.005, 0.001, 0.0005, 0.0001 \}$
    \item $\beta\in\{ 0, 0.95 \}$
\end{itemize}

Our best results were achieved employing the \texttt{sigmoid} activation and $H=20$, on which we had the results shown below in \cref{fig:gridSearch}.

\begin{figure}[h]
    \centering
    \begin{tabular}{|c||c|c|c|c|c|}
            \hline
              $\lambda \backslash \eta$ & $0.1$ & $0.05$ & $0.01$ & $0.005$ & $0.001$\\ \hline\hline
             $0.005$ & 1.461 & 1.493 & 1.465 & 1.461 & 1.728 \\ \hline
             $0.001$ & 1.177 & 1.226 & 1.276 & 1.349 & 1.699 \\ \hline
             $0.0005$ & 1.193 & 1.175 & 1.249 & 1.317 & 1.668 \\ \hline
             $0.0001$ & 1.220 & 1.224 & 1.238 & 1.306 & 1.668 \\ \hline
    \end{tabular}
    \caption{Values of MEE for $H=20$, \texttt{sigmoid} and $\beta=0$}
    \label{fig:gridSearch}
\end{figure}

We run this computationally heavy grid search on multiple machines in a single night: for each choice of hyper-parameters we run the training for 2000 epochs for 10 trials, and then averaged the MEE score on the validation set.

\subsubsection{Chosen models}

As a result of the grid search our fist model had a topology of 10-20-2 with \texttt{sigmoid} activation, and we trained it for 2000 epochs with $\eta=0.1,\lambda=0.001,\beta=0$. Let's call this \emph{network of type A}.

We achieved an average MEE of $1.119$ on our blind test, taken as an average of 10 different trainings of type A networks (to avoid the bias due to the random weight initialization). We can see the learning curve of one of them in \cref{fig:typeA} below.

\begin{figure}[h]
    \centering
    \subfloat[Type A\label{fig:typeA}]{\includegraphics[width=0.55\textwidth]{cup10_one}}
    \subfloat[Type B\label{fig:typeB}]{\includegraphics[width=0.55\textwidth]{cup55}}
    \caption{Learning curves for CUP networks}
\end{figure}


Then we tried an \emph{ensemble} approach: we trained 20 networks of type A and regarded the output of our model as the average of the outputs of the 10 bests (over validation benchmark) trained networks.

In this way we achieved a MEE error on the internal test set of $1.085$.

\bigskip
Our last model uses input preprocessing, and we use a different network architecture, say \emph{type B}.
This NN has a topology of 55-20-2, and is trained with $\eta=0.05,\; \lambda=0.005,\; \beta=0.9$.

The 55 inputs are calculated as follows from the dataset: if the original input is $(x_1,\, \dots,\, x_{10})$, our calculated new vector is $(x_1,\, \dots,x_{10},\, x_1\cdot x_2,\, \dots,\, x_9\cdot x_{10})$, that is obtained from the previous one appending the pairwise products of input variables.

Then we took an ensemble approach and trained 20 networks of type B, and we calculate the output by averaging the best 15 of them.

This ensemble of type B models achieved an MEE of $1.050$ on the internal test set.

The learning curve of one instance of model B can be observed in \cref{fig:typeB}. Values of training and validation MEE of some of the 20 models are in \cref{fig:cup55-table} below.

\begin{figure}
    \centering
    \begin{tabular}{|l|l|l|}
        \hline
        id & Train MEE & Valid MEE \\ \hline
        3 & 1.030 & 1.129 \\ \hline
        5 & 0.968 & 1.250 \\ \hline
        8 & 1.038 & 1.081 \\ \hline
        15 & 1.011 & 1.148 \\ \hline
        20 & 0.987 & 1.327 \\ \hline
    \end{tabular}
    \caption{MEE of some elements of a type B networks ensemble}
    \label{fig:cup55-table}
\end{figure}




%%% Local Variables:
%%% mode: latex
%%% TeX-master: "report"
%%% End:

\section{Conclusions}

\subsection{Final Remarks}
This project gave us the opportunity to have an hands-on experience both in the implementation of a Neural Network and in experimenting model selection and training. We had the possibility to draw inspiration from industry-standard tools and to make design and implementation comparisons between them for our own implementation. We had the possibility to better understand the main theoretical contents of the course through implementation and debugging, for example of the backpropagation algorithm. Similarly, we learnt a lot on how to do model selection and validation from our first attempts to our final version, comparing different strategies and reflecting on how to use techniques like k-fold cross validation and searches results to identify the most promising intervals for hyperparameters. Overall it was a great experience, and we learnt a lot from it.

\subsection{Blind Test Set Results}
Blind test set results are contained in the file \texttt{TheBishops\_ML-CUP22-TS.csv} in the \texttt{results} folder of the project package. Our team name is \textbf{The Bishops}.

\subsection{Agreement}
\noindent\textit{We agree to the disclosure and publication of our names, and of the results with preliminary and final ranking.}

\section*{References}

\bibliographystyle{plain}
\vspace{-1cm}\bibliography{mybib}

%%% Local Variables:
%%% mode: latex
%%% TeX-master: "report"
%%% End:



\end{document}

%%% Local Variables:
%%% mode: latex
%%% TeX-master: t
%%% End: